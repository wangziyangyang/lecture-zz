% !TEX root = ../zz-lecture.tex


\chapter{牛顿定律}

 牛顿运动定律是高中物理的基础,贯穿于高中物理的各个部分。
 在高中物理课程中,我们学习了牛顿运动定律的具体内容和基本应用。
 在未来的两次课中,我们将进一步研究牛顿第二定律(涉及分解加速度、加速度关联等问题)以及在非惯性系中如何使用牛顿第二定律。
 
 \stitle{牛顿第二定律}
 
 
 牛顿第二定律是矢量定律,因此可以分方向使用:
 \begin{equation}
 F_i = ma_i,
 \end{equation}
 其中$ i=1,2,3 $可以理解为力和加速度在给定直角坐标系中的$ x,y,z $分量。
在常规的高考问题中,一般是将外力沿加速度方向和垂直加速度方向进行正交分解,列出分量方程。
在一些复杂问题中,分解加速度列出分量方程可能会使问题更简化。

