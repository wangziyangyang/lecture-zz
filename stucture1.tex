\usepackage{ctex}

\usepackage{amsmath,amssymb,mathtools}
\usepackage{ntheorem}
\usepackage[twoside,top=2.5cm,bottom=2.5cm,left=2.5cm,right=5cm,a4paper,marginparwidth=4cm,marginparsep=0.5cm]{geometry}
\usepackage{physics}
\usepackage{graphicx}
\usepackage{booktabs}
\usepackage{subfigure}
\usepackage{float}
\usepackage{mathrsfs}
\usepackage[colorlinks,linkcolor=blue]{hyperref}
\usepackage{tkz-euclide}
\usepackage{tkz-base}
\usetikzlibrary{math,arrows,backgrounds,scopes,plotmarks,shapes,calc,decorations.pathmorphing,shadows, positioning,fit,petri,intersections,through,backgrounds, shapes.geometric,topaths,automata,mindmap,folding,calendar,chains,circuits,circuits.ee.IEC}
\usetkzobj{all}
\makeatletter
\global\let\tikz@ensure@dollar@catcode=\relax
\makeatother
\usepackage{calc,pifont}
\newcommand\maincolor{cyan}
\usepackage{fancyhdr}
\newcommand\lwidth{17}
\pagestyle{fancy}
%\fancyfoot[r]{\thepage}
\newcommand{\X}{\phantom{X}}


\makeatletter 
\newcommand\figcaption{\def\@captype{figure}\caption} 
\newcommand\tabcaption{\def\@captype{table}\caption} 
\makeatother

\usepackage{caption2}
 \captionsetup{font=footnotesize}
\usepackage{wrapfig}
\usepackage{verbatim}
\usepackage{slashed} 
\usepackage{appendix} 
\usepackage{pdfpages}
\usepackage{wasysym}
\usepackage{setspace}
\usepackage{siunitx}
\usepackage{booktabs}
\usepackage{tagging}
\usepackage{multicol}
%\usepackage{frame}
\usepackage{tabu}
\usepackage{longtable}
\graphicspath{{problem/image/}}

%%%%%%%%%%%%%%%边框文字
\usepackage[explicit]{titlesec}
\usepackage{lmodern}
\usepackage[many]{tcolorbox}
\tcbuselibrary{skins,breakable}
\tcbuselibrary{theorems}


\usepackage{CJKnumb}
\newcommand{\formatsubsection}[1]{
	{\color{\maincolor}\ding{113}\kern0.5em\relax\bf #1}
}
\titleformat{\subsection}[hang]
{}
{}
{0em}
{\formatsubsection{#1}}
\newcommand{\stitle}[1]{\subsection{#1}\label{#1}}



%\theoremstyle{break}
\theorembodyfont{\small}
\theoremindent1cm
\theoremsymbol{\ensuremath{\ast}}
\theoremseparator{}

\newcommand{\formatsection}[1]{
	\begin{center}\begin{tikzpicture}
		\filldraw[draw=\maincolor,fill=\maincolor](0,-.4)--(0,.3)--(.3,.6)--(\lwidth-3.5,.6)--(\lwidth-3.5,-.4)--(0,-.4);
		\node [anchor=west] at(1,.1){\large \color{white} #1};
		\end{tikzpicture}\end{center}
}
\titleformat{\section}[hang]
{\usefont{T1}{qhv}{b}{n}}
{}
{0em}
{\formatsection{#1}}


\newenvironment{post-problem}{
	\newpage\section*{课后练习}
%	\begin{multicols}{2}
	\linespread{1}}{\linespread{1}}
%	\end{multicols}}

\newcounter{example}[chapter]
\newenvironment{example}[1][]{\refstepcounter{example}\par\medskip
	\textbf{\thechapter.\theexample #1} \rmfamily}{\medskip}
%\newtheorem{example}{}[chapter]

\newenvironment{problemfig}{\begin{center}}{ \end{center}}



%\newtcbtheorem[number within=chapter]{eg}{例}
%{breakable,colback=blue!2,colframe=green!25!black,fonttitle=\bfseries,fontupper=\small\CTEXindent,before skip=5mm,code={\singlespacing}}{exa}
\newtcbtheorem[number within=chapter]{app}{例}
{breakable,colback=blue!2,colframe=blue!35!black,fonttitle=\bfseries,fontupper=\small\CTEXindent,fontlower=\small\CTEXindent}{app}

\definecolor{titlebgdark}{RGB}{0,163,243}
\definecolor{titlebglight}{RGB}{191,233,251}


\graphicspath{{image/}} % Specifies the directory where pictures are stored
\renewcommand{\vec}[1]{\vb*{#1}}

\newcommand{\crp}[2]{\vec{#1}\times\vec{#2}}
\newcommand{\dotp}[2]{\vec{#1}\cdot\vec{#2}}
\newcommand{\Ga}[2]{\Gamma^{#1}_{#2}}
\renewcommand{\pb}[2]{\left[{#1},{#2}  \right]_{P.B.}}
\renewcommand{\op}[1]{\^{#1}}
\newcommand{\sch}{Schr\"{o}dinger}
\newcommand{\roma}[1]{\uppercase\expandafter{\romannumeral#1}}
\newcommand{\const}{\text{const.}}
\newcommand{\vdw}{Van der Walls}
\renewcommand{\deg}[1]{#1^\circ}

%\newcommand{\titleform}{第\thechapter 讲}
%%%%%%%%%%%%%%%%%%%%%%%%%%%%%%%%%%%%%%%%%%%%%% chapter format

\titleformat{\chapter}[display]
{\normalfont\huge\bfseries}
{}
{20pt}
{%
	\begin{tcolorbox}[
		enhanced,
		colback=titlebgdark,
		boxrule=0.25cm,
		colframe=titlebglight,
		arc=0pt,
		outer arc=0pt,
		leftrule=0pt,
		rightrule=0pt,
		fontupper=\color{white}\sffamily\bfseries\huge,
		enlarge left by=-1in-\hoffset-\oddsidemargin,
		enlarge right by=-\paperwidth+1in+\hoffset+\oddsidemargin+\textwidth,
		width=\paperwidth,
		left=1in+\hoffset+\oddsidemargin,
		right=\paperwidth-1in-\hoffset-\oddsidemargin-\textwidth,
		top=0.6cm,
		bottom=0.6cm,
		overlay={
			\node[
			fill=titlebgdark,
			draw=titlebglight,
			line width=0.15cm,
			inner sep=0pt,
			text width=1.7cm,
			minimum height=1.7cm,
			align=center,
			font=\color{white}\sffamily\bfseries\fontsize{30}{36}\selectfont
			]
			(chapname)
			at ([xshift=-4cm]frame.north east)
			{\thechapter};
%			\node[font=\small,anchor=south,inner sep=2pt] at (chapname.north)
%			{\MakeUppercase\chaptertitlename};
		}
		]
		#1
	\end{tcolorbox}%
}
\titleformat{name=\chapter,numberless}[display]
{\normalfont\huge\bfseries}
{}
{20pt}
{%
	\begin{tcolorbox}[
		enhanced,
		colback=titlebgdark,
		boxrule=0.25cm,
		colframe=titlebglight,
		arc=0pt,
		outer arc=0pt,
		remember as=title,
		leftrule=0pt,
		rightrule=0pt,
		fontupper=\color{white}\sffamily\bfseries\huge,
		enlarge left by=-1in-\hoffset-\oddsidemargin,
		enlarge right by=-\paperwidth+1in+\hoffset+\oddsidemargin+\textwidth,
		width=\paperwidth,
		left=1in+\hoffset+\oddsidemargin,
		right=\paperwidth-1in-\hoffset-\oddsidemargin-\textwidth,
		top=0.6cm,
		bottom=0.6cm,
		]
		#1
	\end{tcolorbox}%
}
\titlespacing*{\chapter}
{0pt}{0pt}{40pt}
\makeatother

%%%%%%%%%%%%%%%%%%%%%%%%%%%%%%%%%%%%%%%%%%%%

\fancyfoot[c]{}
\renewcommand{\chaptermark}[1]{\markboth{#1}{}}
\fancyfoot[LE,RO]{\begin{tikzpicture}[baseline, every node/.style={minimum size=8mm, anchor=base}]
	\path node at (0,0) [shape=circle, fill=\maincolor!20] (0,0) {\X}
	node at (.8,0) [shape=circle, fill=\maincolor!60] (0,0) {\X}
	node at (1.6,0) [shape=circle, fill=\maincolor] (0,0) { \color{white}\thepage};\end{tikzpicture}}
\fancyhead[L]{\small\it\color{\maincolor}高一$ \cdot $第\thechapter 讲}

\rhead{\small\it\color{\maincolor}\leftmark}



